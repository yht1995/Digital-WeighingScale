\documentclass[16pt,a4paper]{article}

\usepackage{fontspec}
\setmainfont[BoldFont=黑体]{方正新书宋简体}
\XeTeXlinebreaklocale "zh"
\XeTeXlinebreakskip = 0pt plus 1pt minus 0.1pt
\linespread{1.5}

\usepackage{indentfirst}
\setlength{\parindent}{2em}

\usepackage{float}
\usepackage{setspace}
\usepackage[colorlinks,linkcolor=black,anchorcolor=black,citecolor=black]{hyperref}
\usepackage{changepage}

\usepackage{amsmath}
\usepackage{amsfonts}
\usepackage{amssymb}

\usepackage{enumerate}

\usepackage{geometry}
\geometry{left=2.5cm,right=2.5cm,top=2.5cm,bottom=4cm}

\usepackage{fancyhdr}
\pagestyle{fancy}
\lhead{数字体重体脂计}
\author{姚皓天(2013011515) \\ 王勇(2013011521)}
\title{电子技术课程设计 \\ 预习报告}
\date{2015年8月}

\graphicspath{{Figure/}}

\begin{document}
\maketitle
\newpage

\thispagestyle{empty}
\renewcommand\contentsname{\textbf{目录}}
\tableofcontents
\newpage

\section{设计背景}
目前,随着社会的发展、生活水平不断提高,人们越来越关注自己的身体健康。许多人由于工作的压力和不良的饮食习惯,使得身体健康每况愈下,疾病也随之而来,而在这些人群中,患有肥胖和营养不良的病人居多。为方便及时了解自己的体重是否超出或低于标准的体重,我们希望能够及时而准确的对体重进行称量。普通人体秤测量身高和体重的结果都是直接用眼睛观看指标读取的,由于读数的方法各不相同,以及人的心理因素多种原因,使得读取数据的误差过大,故需要将体重值显示出来。\\
但是,事实上,只用体重衡量人的健康程度是不科学的,原因主要有如下几点:
\begin{enumerate}
\item 体重与身高总体成正相关趋势
\item 男性普遍比女性重
\item 身材的好坏与强壮与否不能直接与体重的大小划等号。
\end{enumerate}
这样,体脂率这个名词就进入了我们的视线。体脂率,顾名思义,是指人体内脂肪重量在人体总体重中所占的比例,又称体脂百分数,它反映人体内脂肪含量的多少。正常成年人的体脂率分别是男性15\%-18\%和女性25\%-28\%。若某个人的体脂率过高,并且体重超过正常值的20\%以上就可视为肥胖。体脂过高表明运动不足、营养过剩或有某种内分泌系统的疾病,而且常会并发高血压、高血脂症、动脉硬化、冠心病、糖尿病、胆囊炎等病症;若体脂率过低,低于体脂含量的安全下限,则可能引起功能失调。所以,这也是一个很重要且容易理解的健康指标。\\
当今市面上的智能体重计多如牛毛,也有一些新型的体脂测试仪,然而鲜有将两者的测试同时进行的仪器,我们的选题即为实现此目标而确定。这样实现的多功能人体秤,不但提高了读数的精确度,给人们以直观的效果,同时还能给出体脂率,从而更真实准确的反应健康程度,具有很明显的优势。

\section{实现原理}
\subsection{重力的测量}
利用金属或者半导体的应变效应制作出电阻式传感器。传感器感受到压力时,将导致传感器的敏感元件的电阻发生变化,改变量近似和压力成正比。如果将敏感元件作为一个惠斯通电桥的桥臂,一旦感受到压力,电桥将会失去平衡,从而反映在电桥的输出电压上,根据电路理论和泰勒展开的知识可得,此输出电压近似和电阻改变值成正比,故也就近似和压力成正比。由于电压的输出量很小,这就要利用运放将输出电压进行一定倍数的放大输出。然后将放大的电压信号通过模数转换模块变成数字信号,最后送进行处理输、显示,就可以实现对重力的测量。
\subsection{体脂的测量}
传统上,测量体脂率的方法较为复杂,目前的标准是以DEXA测量为主,利用身体不同组织(矿物质、瘦身体、脂肪)对x光吸收率不同的原理来测量体内脂肪含量的方法,但这样所耗费的时间及费用都相当不经济,而且我们采用较方便的生物电阻测量法测量,(简称BIA),在很短的时间内即可获得颇准确的测量值,适合在家庭中及医师在门诊使用。 \\
BIA测量法的主要原理乃是将身体简单分为导电的体液、肌肉等,以及不导电的脂肪组织。测量时由电极片发出极微小的交变电流经过身体,电流会随着电阻小、传导性能好的体液流传。水分的多少决定了电流通过的通路的宽窄,可测得对应的阻抗值。在人体中,阻抗和电阻的值仅仅相差约$2-3\Omega$,故可用测得的阻抗近似代替人体的电阻值。若脂肪比率高,则所测得的生物电阻较大,反之亦然,BIA就是经由此种机转来做体脂率的测量。
\end{document}
